% Options for packages loaded elsewhere
\PassOptionsToPackage{unicode}{hyperref}
\PassOptionsToPackage{hyphens}{url}
%
\documentclass[
]{article}
\usepackage{amsmath,amssymb}
\usepackage{iftex}
\ifPDFTeX
  \usepackage[T1]{fontenc}
  \usepackage[utf8]{inputenc}
  \usepackage{textcomp} % provide euro and other symbols
\else % if luatex or xetex
  \usepackage{unicode-math} % this also loads fontspec
  \defaultfontfeatures{Scale=MatchLowercase}
  \defaultfontfeatures[\rmfamily]{Ligatures=TeX,Scale=1}
\fi
\usepackage{lmodern}
\ifPDFTeX\else
  % xetex/luatex font selection
\fi
% Use upquote if available, for straight quotes in verbatim environments
\IfFileExists{upquote.sty}{\usepackage{upquote}}{}
\IfFileExists{microtype.sty}{% use microtype if available
  \usepackage[]{microtype}
  \UseMicrotypeSet[protrusion]{basicmath} % disable protrusion for tt fonts
}{}
\makeatletter
\@ifundefined{KOMAClassName}{% if non-KOMA class
  \IfFileExists{parskip.sty}{%
    \usepackage{parskip}
  }{% else
    \setlength{\parindent}{0pt}
    \setlength{\parskip}{6pt plus 2pt minus 1pt}}
}{% if KOMA class
  \KOMAoptions{parskip=half}}
\makeatother
\usepackage{xcolor}
\usepackage[margin=1in]{geometry}
\usepackage{graphicx}
\makeatletter
\def\maxwidth{\ifdim\Gin@nat@width>\linewidth\linewidth\else\Gin@nat@width\fi}
\def\maxheight{\ifdim\Gin@nat@height>\textheight\textheight\else\Gin@nat@height\fi}
\makeatother
% Scale images if necessary, so that they will not overflow the page
% margins by default, and it is still possible to overwrite the defaults
% using explicit options in \includegraphics[width, height, ...]{}
\setkeys{Gin}{width=\maxwidth,height=\maxheight,keepaspectratio}
% Set default figure placement to htbp
\makeatletter
\def\fps@figure{htbp}
\makeatother
\setlength{\emergencystretch}{3em} % prevent overfull lines
\providecommand{\tightlist}{%
  \setlength{\itemsep}{0pt}\setlength{\parskip}{0pt}}
\setcounter{secnumdepth}{-\maxdimen} % remove section numbering
\ifLuaTeX
  \usepackage{selnolig}  % disable illegal ligatures
\fi
\usepackage{bookmark}
\IfFileExists{xurl.sty}{\usepackage{xurl}}{} % add URL line breaks if available
\urlstyle{same}
\hypersetup{
  pdftitle={Business Analytics Project},
  pdfauthor={Raza Rafique},
  hidelinks,
  pdfcreator={LaTeX via pandoc}}

\title{Business Analytics Project}
\author{Raza Rafique}
\date{2024-10-15}

\begin{document}
\maketitle

\section{\texorpdfstring{\textbf{Instructions}}{Instructions}}\label{instructions}

Welcome to the Business Analytics Project! This project will guide you
through a series of deliverables, from selecting a relevant topic to
analyzing data and providing actionable business insights. Below are the
instructions for each deliverable (D1--D4). Make sure to meet the
deadlines for each phase and follow the guidelines for a successful
project outcome.

\begin{center}\rule{0.5\linewidth}{0.5pt}\end{center}

\subsubsection{\texorpdfstring{\textbf{Deliverable 1 (D1): Topic
Selection}}{Deliverable 1 (D1): Topic Selection}}\label{deliverable-1-d1-topic-selection}

\begin{itemize}
\tightlist
\item
  \textbf{Marks}: 20
\end{itemize}

The first step in your business analytics project is to choose a
relevant and impactful topic. Your topic should reflect a real-world
business problem where data-driven decision-making can provide value.

\begin{itemize}
\tightlist
\item
  Select a business area of interest (e.g., marketing, finance,
  operations, supply chain, HR, etc.).
\item
  Clearly define the problem you aim to address (e.g., customer churn,
  inventory management, market segmentation, etc.).
\item
  Briefly justify why the topic is important and how predictive
  analytics can offer insights.
\end{itemize}

\textbf{Submission Requirements}:

\begin{itemize}
\tightlist
\item
  A 1-page summary of your topic selection, including the problem
  statement and its significance to the business.
\item
  Justification of how data analytics will be used to address the
  problem.
\end{itemize}

\begin{center}\rule{0.5\linewidth}{0.5pt}\end{center}

\subsubsection{\texorpdfstring{\textbf{Deliverable 2 (D2): Data Set
Selection/Collection}}{Deliverable 2 (D2): Data Set Selection/Collection}}\label{deliverable-2-d2-data-set-selectioncollection}

\begin{itemize}
\tightlist
\item
  \textbf{Marks}: 30
\end{itemize}

In this phase, you will either collect or select an appropriate dataset
that aligns with your chosen topic.

\begin{itemize}
\tightlist
\item
  Identify a dataset that is suitable for analyzing your chosen business
  problem.
\item
  The dataset should have sufficient data points and variables to
  support meaningful analysis. It can be obtained from public sources
  (e.g., Kaggle, UCI Machine Learning Repository) or collected through
  surveys, business databases, etc.
\item
  Include a brief description of the data (variables, number of
  observations, source).
\end{itemize}

\textbf{Submission Requirements}:

\begin{itemize}
\tightlist
\item
  Submit a 1-page document that describes the dataset (source,
  variables, and why it fits the project).
\item
  Attach the dataset in .csv or Excel format.
\end{itemize}

\begin{center}\rule{0.5\linewidth}{0.5pt}\end{center}

\subsubsection{\texorpdfstring{\textbf{Deliverable 3 (D3): Data
Cleaning, Exploratory Data Analysis (EDA), and Predictive
Analytics}}{Deliverable 3 (D3): Data Cleaning, Exploratory Data Analysis (EDA), and Predictive Analytics}}\label{deliverable-3-d3-data-cleaning-exploratory-data-analysis-eda-and-predictive-analytics}

\begin{itemize}
\tightlist
\item
  \textbf{Marks}: 50
\end{itemize}

In this deliverable, you will clean the dataset, perform EDA, and apply
predictive analytics tools to generate business insights.

\begin{itemize}
\tightlist
\item
  \textbf{Data Cleaning}: Handle missing values, remove duplicates, and
  correct inconsistencies.
\item
  \textbf{Exploratory Data Analysis (EDA)}: Use statistical summaries
  and visualizations (e.g., histograms, box plots) to understand data
  distributions, relationships, and trends.
\item
  \textbf{Predictive Analytics}: Apply predictive models (e.g.,
  regression, classification, clustering) to analyze the data and make
  predictions relevant to the business problem.
\item
  \textbf{Model Training and Evaluation}: Train the model and evaluate
  its performance using metrics like accuracy, RMSE, etc.
\end{itemize}

\textbf{Submission Requirements}:

\begin{itemize}
\tightlist
\item
  Submit a PowerPoint presentation (8-10 slides) including:

  \begin{itemize}
  \tightlist
  \item
    Data cleaning and EDA results.
  \item
    Predictive model approach, findings, and business relevance.
  \item
    Model training, evaluation, and any actionable insights.
  \end{itemize}
\item
  Be prepared to present your findings in class.
\end{itemize}

\begin{center}\rule{0.5\linewidth}{0.5pt}\end{center}

\subsubsection{\texorpdfstring{\textbf{Deliverable 4 (D4): Term Paper --
Business Insights and
Recommendations}}{Deliverable 4 (D4): Term Paper -- Business Insights and Recommendations}}\label{deliverable-4-d4-term-paper-business-insights-and-recommendations}

\begin{itemize}
\tightlist
\item
  \textbf{Marks}: 100
\end{itemize}

The final step is to synthesize your project into a formal term paper.

\begin{itemize}
\tightlist
\item
  Provide a detailed report of your analysis, focusing on key business
  insights from the data and predictive model.
\item
  Discuss the business implications of your findings and offer
  recommendations based on your analysis.
\item
  Highlight any limitations of your analysis and areas for further
  research.
\end{itemize}

\textbf{Submission Requirements}:

\begin{itemize}
\tightlist
\item
  A 8-10 page term paper in APA format, including:

  \begin{itemize}
  \tightlist
  \item
    Introduction: Recap of the problem and project objectives.
  \item
    Data and Methods: Dataset description, data cleaning, and predictive
    analytics methods used.
  \item
    Findings: Key insights from the analysis.
  \item
    Business Recommendations: Practical solutions or actions for the
    business based on your findings.
  \item
    Conclusion: Summary and potential future directions for research or
    business practice.
  \end{itemize}
\end{itemize}

\begin{center}\rule{0.5\linewidth}{0.5pt}\end{center}

\subsubsection{\texorpdfstring{\textbf{General
Guidelines}:}{General Guidelines:}}\label{general-guidelines}

\begin{itemize}
\tightlist
\item
  This is an individual project; ensure that you complete all work
  independently.
\item
  Stay organized and meet deadlines for each deliverable.
\item
  Provide clear, concise, and professional documentation for each phase.
\end{itemize}

Good luck, and I look forward to seeing your analytical insights!

\end{document}
